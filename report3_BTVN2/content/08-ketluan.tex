\newpage
\section{Kết luận}

\paragraph{}{Tổ chức \textbf{bộ vi xử lý} trong \textbf{PC}, \textbf{Server}, \textbf{Mobile} và \textbf{Hệ thống nhúng} phản ánh rõ ràng các yêu cầu và ưu tiên riêng của từng nền tảng. Trong khi PC CPU hướng đến sự cân bằng và đa năng, Server CPU tập trung vào độ tin cậy và sức mạnh xử lý song song. Mobile CPU (SoC) ưu tiên tuyệt đối cho hiệu quả năng lượng và tích hợp cao độ. Cuối cùng, CPU/MCU trong hệ thống nhúng được tối ưu cho các tác vụ chuyên biệt với chi phí và mức tiêu thụ năng lượng thấp nhất.}

\paragraph{}{Hiểu rõ những khác biệt này giúp chúng ta lựa chọn và thiết kế các hệ thống hiệu quả hơn, đáp ứng đúng nhu cầu sử dụng. Sự phát triển không ngừng của công nghệ vi xử lý hứa hẹn sẽ tiếp tục mang đến những cải tiến vượt bậc về hiệu năng, tính năng và hiệu quả năng lượng cho tất cả các nền tảng trong tương lai.}

\paragraph{Xu hướng chung:}

\begin{itemize}
    \item \textbf{Tăng số lượng lõi}: Để cải thiện khả năng xử lý đa nhiệm và song song.
    
    \item \textbf{Chuyên biệt hóa}: Tích hợp các đơn vị xử lý chuyên dụng như NPU cho AI, ISP cho hình ảnh.

    \item \textbf{Kiến trúc Chiplet:} Thay vì một chip lớn duy nhất (monolithic), các nhà sản xuất đang hướng tới việc kết hợp nhiều chip nhỏ (chiplets) trên cùng một đế, giúp tăng hiệu suất, giảm chi phí và linh hoạt hơn trong thiết kế.

    \item \textbf{Tập trung vào hiệu quả năng lượng}: Ngay cả với PC và Server, việc tối ưu hiệu năng trên mỗi Watt điện tiêu thụ ngày càng quan trọng.
    
    \item \textbf{Bảo mật phần cứng}: Tích hợp các tính năng bảo mật sâu hơn vào kiến trúc vi xử lý.
\end{itemize}

\pagebreak